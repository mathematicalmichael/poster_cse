% Gemini theme
% https://github.com/anishathalye/gemini

\documentclass[final]{beamer}

% ====================
% Packages
% ====================

\usepackage[T1]{fontenc}
\usepackage{lmodern}
\usepackage[size=custom,width=75,height=100,scale=1.0]{beamerposter}
\usetheme{gemini}
\usecolortheme{labsix}
\usepackage{graphicx}
\usepackage{booktabs}
\usepackage{tikz}
\usepackage{pgfplots}
\pgfplotsset{compat=1.15}

% ====================
% Lengths
% ====================

% If you have N columns, choose \sepwidth and \colwidth such that
% (N+1)*\sepwidth + N*\colwidth = \paperwidth
\newlength{\sepwidth}
\newlength{\colwidth}
\setlength{\sepwidth}{0.0333\paperwidth}
\setlength{\colwidth}{0.45\paperwidth}

% Spacing / Formatting
\newcommand{\separatorcolumn}{\begin{column}{\sepwidth}\end{column}}
\newcommand{\itembox}[1]{\item {\makebox[7cm]{#1 \hfill}}}

% Standard Probability
\DeclareMathOperator{\E}{\mathbb{E}}
\newcommand{\PP}{\mathbb{P}}
\newcommand{\RR}{\mathbb{R}}

% Data-Consistent Inversion Framework

\DeclareMathOperator*{\param}{\lambda}
\DeclareMathOperator*{\pspace}{\Lambda}
\newcommand{\pmeas}{\mu_{\pspace}}
\newcommand{\pborel}{\mathcal{B}_{\pspace}}

\DeclareMathOperator*{\initialP}{\PP^{in}}
\DeclareMathOperator*{\initial}{\pi^{in}}
\DeclareMathOperator*{\updatedP}{\PP^{up}}
\DeclareMathOperator*{\updated}{\pi^{up}}

\DeclareMathOperator*{\noise}{\xi}
\DeclareMathOperator*{\nspace}{\Xi}

\newcommand{\obs}{\boldsymbol{o}}
\newcommand{\data}{\boldsymbol{d}}
\DeclareMathOperator*{\dspace}{\mathcal{D}}
\newcommand{\dmeas}{\mu_{\dspace}}
\newcommand{\dborel}{\mathcal{B}_{\dspace}}

\DeclareMathOperator*{\lam}{\left ( \param \right ) }
\DeclareMathOperator*{\qoi}{Q \lam }
\DeclareMathOperator*{\qlam}{\left ( \qoi \right ) }
\DeclareMathOperator*{\M}{\mathcal{M}}

\DeclareMathOperator*{\observedP}{\PP^{ob}}
\DeclareMathOperator*{\observed}{\pi^{ob}}
\DeclareMathOperator*{\predictedP}{\PP^{pr}}
\DeclareMathOperator*{\predicted}{\pi^{pr}}

\DeclareMathOperator*{\dciP}{\updatedP = \initialP \frac{\observedP}{\predictedP}}
\DeclareMathOperator*{\dciD}{\updated = \initial \frac{\observed}{\predicted}}
\DeclareMathOperator*{\dci}{\updated\lam = \initial\lam \frac{\observed\qlam}{\predicted\qlam}}

% ====================
% Title
% ====================

\title{Push-forward Measures for Parameter Identification under Uncertainty}

\author{Michael Pilosov \inst{1}}

\institute[shortinst]{\inst{1} University of Colorado: Denver}

% ====================
% Body
% ====================

\begin{document}

\begin{frame}[t]
\begin{columns}[t]
\separatorcolumn

\begin{column}{\colwidth}

  \begin{block}{Notation}
\large
    \begin{itemize}
        \itembox{$\pspace \subset \RR^P$} Parameter Space
        \itembox{$\obs: \pspace \to \mathcal{O} \subset \RR^D$} Observables
        \itembox{$\nspace \subset \RR^D$} Noise Space
        \itembox{$\param^\dagger\in\pspace$} True Parameter
        \itembox{$\data(\noise) \subset \RR^D$} Possible Data, where $d_i(\noise) = \obs_i(\param^\dagger) + \xi_i$
        \itembox{$\noise^\dagger\in\nspace$} Noise in Measurements
        \itembox{$ \sigma $} Variance of Noise
        \itembox{$\data^\dagger\in\RR^D$} Observed Data, where $\data^\dagger = \data(\noise^\dagger)$
    \end{itemize}

\end{block}

\begin{block}{Quantity of Interest Map}
\centering
\large
    \emph{Defines a (functional) relationship between predictions and data}

        $$Q \left (\param, \noise \right ) = F \left ( \obs(\param), \data(\noise) \right ) $$
        $$Q \left (\param \right ) = F \left ( \obs(\param), \data^\dagger \right )$$
        $$Q \left ( \pspace, \nspace \right ) =: \dspace_t \subset \RR$$
        $$Q \left ( \pspace \right ) =: \dspace_c \subset \dspace_t$$

    \begin{figure}
        \includegraphics[height=20cm]{figure}
    \end{figure}

\large
    \emph{How do conditionals of $\nspace$ compare to the joint density?}

\end{block}


\end{column}

\separatorcolumn

\begin{column}{\colwidth}


  \begin{block}{Framework}
\large
    \begin{itemize}
        \itembox{$ \initialP, \; \initial $} Initial
        \itembox{$ \observedP, \; \observed $} Observed
        \itembox{$ \predictedP, \; \predicted $} Predicted (\bf{Push-forward of Initial})
        \itembox{$ \updatedP, \; \updated $} Updated
    \end{itemize}
\Large
    \heading{Updating Initials with Observations and Predictions}
\begin{equation*}
        \dciP, \quad \dciD
    \end{equation*}

\end{block}




  \begin{block}{The Observed Distribution}
\centering
\Large
    $Q(\param^\dagger, \noise) \sim \observed$ when we allow $\noise$ to vary over $\nspace$.
\vspace{1cm}
    \begin{table}
      \centering
      \begin{tabular}{c c c}
        %\toprule
        \textbf{$F(\obs, \data^\dagger)$} & \textbf{$\noise$} & {$\observed$} \\
        \midrule
        $\frac{1}{\sigma\sqrt{D}} \large\sum \left( \obs_i\lam - \data_i^\dagger \right)$ & $\noise \sim L^2$ & $N(0,1)$ \\
        $\frac{1}{\sigma^2} \large\sum \left ( \obs_i\lam - \data_i^\dagger \right)^2$ & $\noise \sim N(0,\sigma^2)$ & $\chi^2 (D)$ \\
        $\frac{1}{\sigma^2 D} \large\sum \left ( \obs_i\lam - \data_i^\dagger \right)^2$ & $\noise \sim N(0,\sigma^2)$ & $\Gamma \left ( D/2, D/2) \right )$ \\
        \dots & \dots & \dots \\
        \bottomrule
      \end{tabular}
      \caption{\large Choices of $F$ and associated $\observed$ for problem, where $\data_i^\dagger = \data(\noise_i^\dagger)$.}
    \end{table}

\end{block}
\vspace{-1cm}


    How does any individual conditional of $\nspace$ compare to the joint density?


  \begin{block}{Examples}

    \begin{figure}
        \includegraphics[width=30cm]{error_id_map}
    \end{figure}
    \begin{figure}
        \includegraphics[width=30cm]{predicted}
    \end{figure}

\end{block}



  \begin{block}{References}
    \centering
    \begin{figure}
        \includegraphics{ref1}
    \end{figure}
   
    %\nocite{*}
    %\footnotesize{\bibliographystyle{plain}\bibliography{poster}}

  \end{block}

\end{column}

\separatorcolumn
\end{columns}
\end{frame}

\end{document}
